\documentclass[]{article}
\usepackage{amsmath,amssymb}
\usepackage{lmodern}
\usepackage{iftex}
\usepackage{ragged2e}
\usepackage{stackengine}
\usepackage{dirtytalk}
\usepackage{graphicx}
% \graphicspath{{./Images/}}
\usepackage{array}   % for \newcolumntype macro
\newcolumntype{L}{>{$}l<{$}}
\hbadness=99999

\date{March 2023}
\author{ID: 1747}
\title{Cryptology Packet Ch 5.1}

\begin{document}

\maketitle

\begin{enumerate}
    \item At this point, you should stop and do the Google Sheets activity, “Modular Inverses”. Try to answer this question: Why must E be relatively prime to $(p-1)(q-1)$?
    \\\\From Packet: If E has factors in common with $(p-1)(q-1)$, then it will not have a multiplicative inverse in $\mathbb{Z}_{(p-1)(q-1)}$. That is, we wouldn’t be able to find the corresponding decoding number, $D$.
    
    \item Let $p=3$ and let $q=5$. Then $(p-1)(q-1)=8$ and $pq=15$. Suppose that we choose $E$ to be $3$. Find $D$. Remember that $\mathbb{Z}_8 = \{0,1,2,3,4,5,6,7\}$.
    \\\\$D$ is defined as the multiplicative inverse of $E$ in $\mathbb{Z}_{(p-1)(q-1)}$. So if $E=3$, then we can find the multiplicative inverse of $E$ while being in $\mathbb{Z}_8$. 
    \\$3*D \equiv 1 \mod 8$
    \\$3*3 \equiv 1 \mod 8$
    \\Therefore, D will be $3$.

    \item Sticking with the same $p, q, $ and $E$ (and therefore the same D), complete the table below using the rules of $\mathbb{Z}_{15}$. Remember that you don't simplify exponents according to the rules of mod numbers: exponents are regular integers. What do you notice about the entries of the last row?
    \\\\ 
    \begin{tabular}{| L || L | L | L | L | L | L | L | L | L | L | L | L | L | L}
        \hline
        M \mod pq & 1 & 2 & 3 & 4 & 5 & 6 & 7 & 8 & 9 & 10 & 11 & 12 & 13 & 14 \\
        \hline
        \hline
        M^E \mod pq & 1 & 8 & 12 & 4 & 5 & 6 & 13 & 2 & 9 & 10 & 11 & 3 & 7 & 14 \\
        \hline
        M^{ED} \mod pq & 1 & 2 & 3 & 4 & 5 & 6 & 7 & 8 & 9 & 10 & 11 & 12 & 13 & 14 \\
        \hline
    \end{tabular}

    \item At this point, you should stop and do the Google Sheets activity, \say{RSA Cryptosystem Crux Theorem}. State a conjecture based on your observations from the activity
    \\\\From the google sheet activity, I found that $M \equiv M^{ED} \mod pq$ where M is a number in $\mathbb{Z}_{pq}$

    \item Find the primes $p$ and $q$ if $pq=14,647$ and $\phi(pq)=14400$
    \\\\$\phi(pq)=\phi(p)*\phi(q)=(p-1)(q-1)=pq-q-p+1$
    \\$14400=14647-q-p+1$
    \\$-248=-q-p$
    \\$p+q=248$
    \\$q=248-p$
    \\$pq=14647$
    \\$p(248-p)=14647$
    \\$248p-p^2=14647$
    \\$p^2-248p+14647=0$
    \\$(p-97)(p-151)=0$
    \\$p=97,151$
    \\$p=97,q=151$

    \item Prove Theorem 5.2 based on what you have already learned (perhaps in a previous section).
    \\\\$M^{\phi(pq)} \equiv 1 \mod pq$(Euler's theorem) 
    \\$\phi(pq)=\phi(p)*\phi(q)=(p-1)(q-1)$
    \\$M^{\phi(pq)}=M^{(p-1)(q-1)}$
    \\$M^{(p-1)(q-1)} \equiv 1 \mod pq$

    \item Prove Theorem 5.3 based on what you have already learned.
    \\\\$M^{(p-1)(q-1)} \equiv 1 \mod pq$
    \\$M^{k(p-1)(q-1)} \equiv 1^k \mod pq$
    \\$M^{1+k(p-1)(q-1))} \equiv 1^k*M \mod pq$
    \\$M^{1+k(p-1)(q-1))} \equiv M \mod pq$

    \item Prove Theorem 5.1 (RSA Cryptosystem Crux) based on what you have already learned.
    \\\\From theorem 5.4, $D*E \equiv 1 \mod (p-1)(q-1)$ which is also $D*E \equiv 1+k(p-1)(q-1)$ where $k$ is a positive integer. 
    \\\\If $M$ is a number in $\mathbb{Z}_{pq}$ then $M^{ED} = M^{1+k(p-1)(q-1)}$. 
    \\\\From theorem 5.3, $M^{1+k(p-1)(q-1)} \equiv M \mod pq$ so $M^{ED} \equiv M \mod pq$

    
\end{enumerate}


\end{document}
