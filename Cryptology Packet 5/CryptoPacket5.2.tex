\documentclass[]{article}
\usepackage{amsmath,amssymb}
\usepackage{lmodern}
\usepackage{iftex}
\usepackage{ragged2e}
\usepackage{stackengine}
\usepackage{dirtytalk}
\usepackage{graphicx}
% \graphicspath{{./Images/}}
\usepackage{array}   % for \newcolumntype macro
\newcolumntype{L}{>{$}l<{$}}
\hbadness=99999

\date{April 2023}
\author{ID: 1747}
\title{Cryptology Packet Ch 5.2}

\begin{document}

\maketitle

\begin{enumerate}
    \item Suppose that Alice's secret message is the number $M=2$. What number does she send Bob? Describe in your own words how you found this number.
    \\\\$M^{E} \mod 291$
    \\$2^{5} \mod 291$
    \\$32 \mod 291$
    \\$32 $
    \\\\Using the description from the problem, we know that Alice encodes her message by doing $M^E$ and then modding it by 291 as the answer is to be computed in $\mathbb{Z}_{291}$. Therefore, we can use $M^{E} \mod 291$ to find the number that Alice sent to Bob, which is 32. 

    \item Suppose that Alice’s secret message is the number $M=150$. What number does she send Bob? If you try to do this directly, your calculator might overflow, but there is a way to avoid this problem (and these sorts of computational shortcuts are important in practical implementations of RSA). Describe how you found the number in your own words.
    \\\\$M^{E} \mod 291$
    \\$150^{5} \mod 291$
    \\$150^{2}*150^{2}*150^{1} \mod 291$
    \\$93 * 93 * 150 \mod 291$
    \\$1297350 \mod 291$
    \\$72$
    \\\\I followed the same encryption method as the previous problem but separated the $150^5$ into $150^2 * 150^2 *150^1$. Then, I found $150^2 \mod 291$ to be $93$ and found that $93*93*150 \mod 291$ is $72$.

    \item Bob can now decode Alice's message:
        \begin{enumerate}
            \item Verify that $77$ is indeed the multiplicative inverse of $5$ in $\mathbb{Z}_{192}$. Explain in your own words how you know that you are correct.
            \\\\$77*5 \mod 192$
            \\$385 \mod 192$
            \\$1 \mod 192$
            \\\\First, we know that a multiplicative inverse for $5 \mod 192$ exists because they are relatively prime. Modulo multiplicative inverse of two numbers, $x$ and $y$, will be $x*y \equiv 1 \mod m$. $77$ is the multiplicative inverse of $5$ because we get $1 \mod 192$ when we do $77*5 \mod 192$.

            \item Using the RSA Cryptosystem Crux Theorem, explain how Bob can use the number $D$ to decode Alice’s encoded message $M^E$ and recover her original message $M$.
            \\\\$M^{E^D} \equiv M \mod pq$
            \\\\Therefore, Bob can decode Alice's encoded message $M^E$ and recover her message $M$ by finding $M^{E^D} \mod pq$.
            \\\\For example, using $M=2$, $E=5$, $D=77$, $pq=291$, $(p-1)(q-1)=192$
            \\$M^{E^D} \equiv M \mod pq$
            \\$2^{5^{77}} \mod pq$
            \\$2^{5^{77}} \mod 291$
            \\$2^{5*77} \mod 291$
            \\$2^{385} \mod 291$
            \\$2^{1}*2^{384} \mod 291$
            \\$2^{1}*2^{192*2} \mod 291$
            \\$2^{1+2(192)} \mod 291$
            \\$2 \mod 291$ (Using theorem 5.3)
            
        \end{enumerate}


    
\end{enumerate}


\end{document}
