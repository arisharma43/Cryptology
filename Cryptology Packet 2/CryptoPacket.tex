\documentclass[]{article}
\usepackage{amsmath,amssymb}
\usepackage{lmodern}
\usepackage{iftex}
\usepackage{ragged2e}
\usepackage{stackengine}
\usepackage{dirtytalk}
\usepackage{graphicx}
% \graphicspath{{./Images/}}
\usepackage{array}   % for \newcolumntype macro
\newcolumntype{L}{>{$}l<{$}}
\hbadness=99999

\date{January 2023}
\author{ID: 1747}
\title{Cryptology Packet Pt2}

\begin{document}

\maketitle

\begin{enumerate}
    \item Consider the integers modulo $6$
    \\\\a. Construct a table for addition modulo $6$
    \\\\ 
    \begin{tabular}{| L || L | L | L | L | L | L |}
        \hline
        + & 0 & 1 & 2 & 3 & 4 & 5 \\
        \hline
        \hline
        0 & 0 & 1 & 2 & 3 & 4 & 5 \\
        1 & 1 & 2 & 3 & 4 & 5 & 0 \\
        2 & 2 & 3 & 4 & 5 & 0 & 1 \\
        3 & 3 & 4 & 5 & 0 & 1 & 2 \\
        4 & 4 & 5 & 0 & 1 & 2 & 3 \\
        5 & 5 & 0 & 1 & 2 & 3 & 4 \\
        \hline
    \end{tabular}
    \\\\b. Construct a table for subtraction modulo $6$
    \\\\
    \begin{tabular}{| L || L | L | L | L | L | L |}
        \hline
        - & 0 & 1 & 2 & 3 & 4 & 5 \\
        \hline
        \hline
        0 & 0 & 5 & 4 & 3 & 2 & 1 \\
        1 & 1 & 0 & 5 & 4 & 3 & 2 \\
        2 & 2 & 1 & 0 & 5 & 4 & 3 \\
        3 & 3 & 2 & 1 & 0 & 5 & 4 \\
        4 & 4 & 3 & 2 & 1 & 0 & 5 \\
        5 & 5 & 4 & 3 & 2 & 1 & 0 \\
        \hline
    \end{tabular}
    \\\\c. Construct a table for multiplication modulo $6$
    \\\\
    \begin{tabular}{| L || L | L | L | L | L | L |}
        \hline
        * & 0 & 1 & 2 & 3 & 4 & 5 \\
        \hline
        \hline
        0 & 0 & 0 & 0 & 0 & 0 & 0 \\
        1 & 0 & 1 & 2 & 3 & 4 & 5 \\
        2 & 0 & 2 & 4 & 0 & 2 & 4 \\
        3 & 0 & 3 & 0 & 3 & 0 & 3 \\
        4 & 0 & 4 & 2 & 0 & 4 & 2 \\
        5 & 0 & 5 & 4 & 3 & 2 & 1 \\
        \hline
    \end{tabular}

    \item Which decimal digits occur as the final digit of a fourth power of an integer?
    \\\\ $0,1,5,6$. If a number is raised to the power of $4$, it will be divisible by 10 except for the final digit. To check the final digit of a fourth power of an integer, we can check the $10$ cases for the final digit ($0-9$). Finally, to find the possible options, we take mod $10$ of each final digit to the power of $4$. 
    \\\\ Proof:
    \\ $0^4 \text{ mod } 10 = 0$
    \\ $1^4 \text{ mod } 10 = 1$
    \\ $2^4 \text{ mod } 10 = 6$
    \\ $3^4 \text{ mod } 10 = 1$
    \\ $4^4 \text{ mod } 10 = 6$
    \\ $5^4 \text{ mod } 10 = 5$
    \\ $6^4 \text{ mod } 10 = 0$
    \\ $7^4 \text{ mod } 10 = 1$
    \\ $8^4 \text{ mod } 10 = 6$
    \\ $9^4 \text{ mod } 10 = 1$

    \item Compute the number $k$ in $\mathbb{Z}_{12}$ such that $37^{453} \equiv \text{ mod } 12$. Explain.
    \\\\ We need to find a number $k$ that is from $0-11$. We can use a multiple of $12$ to find what $k$ is. For example, we know that $37 \equiv 1 \text{ mod } 36$. So, $37^{453} \equiv 1^{453} \text{ mod } 36$ will also be true. Therefore, $37^{453} \equiv 1 \text{ mod } 12$ and $k=1$.

    \item Compute the number $k$ in $\mathbb{Z}_{7}$ such that $2^{50} \equiv \text{ mod } 7$ without using a computer.
    \\\\ We need to find a number $k$ that is from $0-6$. There is a pattern that repeats that can be seen as $2^0 \equiv 1 \text{ mod }7$, $2^1 \equiv 2 \text{ mod }7$, $2^2 \equiv 4 \text{ mod }7$, $2^3 \equiv 1 \text{ mod }7$. Since $50$ is equivalent to $3*16+2$, $2^{50}=2^{3*16+2}=2^{3*16}*2^2=(1)^{16} * 4 = 4$. Therefore, $k=4$.

    \item Compute the number $k$ in $\mathbb{Z}_{12}$ such that $39^{453} \equiv \text{ mod } 12$ without using a computer.
    \\\\ We need to find a number $k$ that is from $0-11$. There is a pattern that repeats that can be seen as $3^1 \equiv 3 \text{ mod }12$, $3^2 \equiv 9 \text{ mod }12$, $3^3 \equiv 3 \text{ mod }12$, $3^4 \equiv 9 \text{ mod }12$.
    \\\\$3^{453} \equiv k \text{ mod } 12$
    \\$k \equiv 3^{453} \text{ mod } 12$
    \\$k \equiv 3^{(3*151)} \text{ mod } 12$
    \\$k \equiv 3^{(3)(151)} \text{ mod } 12$
    \\$k \equiv 3^{151} \text{ mod } 12$
    \\$k \equiv 3^{3*50+1} \text{ mod } 12$
    \\$k \equiv 3^{3(50)}*3 \text{ mod } 12$
    \\$k \equiv 3^{(51)} \text{ mod } 12$
    \\$k \equiv 3^{(17*3)} \text{ mod } 12$
    \\$k \equiv 3^{(17)} \text{ mod } 12$
    \\$k \equiv 3^{3*5+2} \text{ mod } 12$
    \\$k \equiv (3^{3})^5*3^{2} \text{ mod } 12$
    \\$k \equiv (3^{7}) \text{ mod } 12$
    \\$k \equiv (3^{3*2+1}) \text{ mod } 12$
    \\$k \equiv (3^{3})^2*3^{1} \text{ mod } 12$
    \\$k \equiv (3^{2}*3) \text{ mod } 12$
    \\$k \equiv (3^{3}) \text{ mod } 12$
    \\$k \equiv (3) \text{ mod } 12$
    \\$3^{453} \equiv 3 \text{ mod } 12$
    \\\\Instead of doing all the work above, since $453$ is odd, it will follow the odd pattern, $3^1 \equiv 3 \text{ mod }12$, $3^3 \equiv 3 \text{ mod }12$, and so on so that $k=3$

    \item Find the numbers in $\mathbb{Z}_{47}$ that are congruent to each of the following without using a computer.
    \\\\a. $2^{32}$
    \\\\$2^{32} \equiv k \mod 47$
    \\$k \equiv 2^{32} \mod 47$
    \\$k \equiv 2^{16*2} \mod 47$
    \\$k \equiv 256^4 \mod 47$
    \\$k \equiv 21^4 \mod 47$
    \\$k \equiv 21^4 \mod 47$
    \\$k \equiv 3^4 * 7^4 \mod 47$
    \\$k \equiv 81 * 7^4 \mod 47$
    \\$k \equiv 34 * 49^2 \mod 47$
    \\$k \equiv 34 * 2^2 \mod 47$
    \\$k \equiv 34 * 4 \mod 47$
    \\$k \equiv 136 \mod 47$
    \\$2^{32} \equiv 42 \mod 47$
    \\\\b. $2^{47}$
    \\\\From Fermat's little theorem, $a^p \equiv a \mod p$. Therefore, $2^{47} \equiv 2 \mod 47$
    \\\\c. $2^{200}$
    \\\\$2^{200} \equiv k \mod 47$
    \\$k \equiv 2^{200} \mod 47$
    \\$k \equiv 2^{47*4}*2^{12} \mod 47$
    \\$k \equiv 2^{4}*2^{12} \mod 47$
    \\$k \equiv 2^{4}*2^{4*3} \mod 47$
    \\$k \equiv 16*16^3 \mod 47$
    \\$k \equiv 16^4 \mod 47$
    \\$k \equiv 256*256 \mod 47$
    \\$k \equiv 21*21 \mod 47$
    \\$k \equiv 49*9 \mod 47$
    \\$k \equiv 2*9 \mod 47$
    \\$k \equiv 18 \mod 47$
    \\$2^{200} \equiv 18 \mod 47$

    \item Find the canonical residue congruent to each of the following without using a computer.
    \\\\a. $3^{10} \mod 11$
    \\\\$3^{10} \mod 11$
    \\$3^{11} * 3^{-1} \mod 11$
    \\$3 * (1/3)  \mod 11$
    \\$1 \mod 11$
    \\$1$
    \\\\b. $2^{12} \mod 13$
    \\\\$2^{12} \mod 13$
    \\$2^{13} * 2^{-1} \mod 13$
    \\$2 * 2^{-1} \mod 13$
    \\$2 * (1/2) \mod 13$
    \\$1 \mod 13$
    \\$1$
    \\\\c. $5^{16} \mod 17$
    \\\\$5^{16} \mod 17$
    \\$5^{17} * 2^{-1} \mod 17$
    \\$5 * 5^{-1} \mod 13$
    \\$5 * (1/5) \mod 13$
    \\$1 \mod 13$
    \\$1$
    \\\\d. $3^{22} \mod 23$
    \\\\$3^{22} \mod 23$
    \\$3^{23} * 3^{-1} \mod 23$
    \\$3 * 3^{-1} \mod 23$
    \\$3 * (1/3) \mod 23$
    \\$1 \mod 23$
    \\$1$
    \\\\e. Make a conjecture based on the congruences in this problem
    \\\\For a number $a$ in the canonical complete residue system modulo $p$, $a^{p-1} \equiv 1 ( \mod p)$ 
    
\end{enumerate}


\end{document}
